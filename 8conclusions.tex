%!TEX root = thesis.tex

\chapter{Conclusions}
\label{chapter:conclusions}

In this thesis, we studied the effects of a reusable web geovisualization tool on the effort needed for building visualizations, and on the quality of visualizations. In order to do this, we studied software reuse and geovisualizations, implemented a reusable web geovisualization tool and evaluated the tool against visualizations built without it. Our principal findings were that the tool enables visualizers to build \emph{more effective} visualizations \emph{more efficiently} at least in certain situations.

Geographical data is data with a geospatial dimension, such as Point of Interest with location data as coordinates. When such data is visualized based on the geospatial dimension, the resulting visualization is called \emph{geographical visualization} (or \emph{geovisualization}). Most typically, geographical visualization is done using a map using the process called \emph{thematic mapping}. In the past, thematic maps were predominantly made by cartographers. However, recently, the advent of web-based mapping tools has enabled non-cartographers to create various map visualizations. Currently, it is estimated that the majority of web map visualizations are made by laymen with no education related to cartography.

% While the definition of visualization is disputed, visualizations can be measured in several ways. According to \citet{tufte_visual_1986}, visualizations should, above all, show the data consistently. This is a valuable guideline, but too vague for exact measurements. For more concrete guidelines, \citet{zuk_heuristics_2006} present a number of heuristics for evaluating visualizations. The heuristics consist mainly of psychological guidelines applied for visualization use.

% Thematic maps can be created using a number of \emph{mapping methods}. Choropleth map is the most frequently used mapping method. Choropleth maps consist of predefined enumeration areas (e.g., countries or municipalities) for which a uniform visualizable property (e.g., median income) can be determined. Isarithmic maps are visualizations depicting smooth or continuous phenomena, such as elevation. In isarithmic map, phenomena are typically visualized using gradients, often accompanied by \emph{contour lines}. Dot maps and proportional symbol maps are used to display phenomena related to point location, typically with dots or markers placed to the locations. Additional widely used mapping methods include dasymetric maps, cartograms and flow maps. Additionally, a single geographic visualization can consist of several different mapping methods.

% Software reuse has great potential. When applied appropriately, it typically results in a considerably reduced need for effort and higher quality. However, the benefits of reuse are difficult to measure. Typically, software reuse benefits primarily on long-term; on short-term, it may be perceived as disadvantageous. Several methods exist for creating reusable software. Scavenging is an \emph{ad hoc} reuse method in which code and other assets are copied from code base to another as is. Source code components are built as reusable pieces of software with explicit interfaces for integration. Application generators are programs designed to output another program according to high-level specifications. Software frameworks combine software components and programming patterns to provide a comprehensive approach for reuse. Typically, several of the methods can be used together.

Before this work, apart from general-purpose mapping libraries, practically no libraries exist for building geovisualizations on the web. General-purpose mapping libraries such as Google Maps API support building visualizations to some degree. However, these libraries are of too low abstraction level to allow building more complex visualizations efficiently. Moreover, as the libraries are not designed for building visualizations, they do not encourage building effective visualizations. Therefore, it is both laborious and difficult to build effective map visualizations with general-purpose mapping tools. Moreover, research on the effect of software reuse on the quality of visualization is scant.

During this work, we implemented Thematic.js, a reusable visualization application for the web. The application can be used independently as a single-page application, or as a part of other JavaScript applications. The application is designed to support the most frequently used thematic mapping methods and relevant utility functionality. Additionally, the application architecture is designed to be modular for effective extensibility.

We evaluated the implemented application by defining several use cases depicting typical geovisualization use. Then, we implemented the cases using Thematic.js, and reference implementations for comparison using Leaflet.js, a general-purpose mapping library. These implementations were then compared using several metrics for software development effort. Additionally, we evaluated the Thematic.js implementations according to the heuristics of \citet{zuk_heuristics_2006} and objectives of \citet{schlichtmann_visualization_2002}.

The evaluation results indicated that using Thematic.js, building geographic visualizations is significantly less laborious. In the test cases evaluated, implementations built with Thematic.js required an average of 7 \% of the effort needed for the reference implementation. Moreover, Thematic.js implementations used only 14 \% of physical and 20 \% of logical lines of code, introduced 13 \% of the complexity and 37 \% difficulty when compared to reference implementations. The results are statistically significant using the significance level of 1 \%.

Additionally, we deemed using Thematic.js beneficial regarding 16 of 35 visualization effectiveness guidelines (heuristics and objectives), with only 1 of 35 guidelines deemed negative. This indicates that Thematic.js is beneficial to the effectiveness of the visualization at least in certain cases.

Thus, we can conclude the findings in relation to the research questions selected:

\begin{enumerate}
	\item[RQ1] How does a reusable software system affect the \emph{efficiency} of building geographical visualizations?
	\item[A1] According to our findings, a reusable software system such as Thematic.js increases the efficiency of building geographical visualizations.
	\item[RQ2] Can a reusable software system enable creating \emph{effective} geographical visualizations?
	\item[A2] According to our findings, a reusable software system such as Thematic.js encourages visualizers to create effective geographical visualizations.
\end{enumerate}

While the results look highly promising, reliable effort measurement is incredibly hard. Therefore, we presented a number of concerns regarding the validity of results. First, the correct level of inclusion of reusable parts of a software system for measurements is disputed. We decided to exclude the reusable parts. Measurements with reusable parts included would likely yield more negative results regarding the effort needed. Second, the metrics do not take into consideration different experience levels of visualizers and thus the results may vary greatly depending on the visualizer. Third, while the use cases were selected according to approximate usage of visualization methods, using different methods would probably yield highly different results.

