%!TEX root = thesis.tex

\chapter{Introduction}
\label{chapter:intro}

The introduction in itself is rarely very long; two to five pages often
suffice.

\section{Problem statement}

Currently, there are several libraries available for displaying maps and simple visualizations \fixme{add references to the libraries. GMaps, Leaflet, OpenLayers etc.}. However, none of the mainstream libraries is of sufficiently high abstraction level for building map visualizations effectively, resulting in the need for writing \emph{boilerplate} code that does not directly contribute to the visualization.

We plan to evaluate means to make creating map visualizations for the web more efficient by building a higher abstraction level software framework for map visualizations. This framework should provide the structure for creating the visualization as well as common web application features needed in modern web applications.

% TODO remove
% \section{Helpful hints}

% Read the information from the university master's thesis
% pages~\citep{ThesisInstructions} before starting the thesis.  You
% should also go through the thesis grading
% instructions~\citep{ThesisGrading} together with your instructor and/or
% supervisor in the beginning of your work.

\section{Structure of the Thesis}
\label{section:structure} 

Chapter 1 (this introduction) presents the motivation for this thesis as well as the problem statement. Chapter 2 describes the background of the work. In particular, the chapter describes how to visualize geographical data and the essence of web software frameworks. Chapter 3 presents the web technology, standards and other needed material for building a web visualization as well as describing some existing map visualizations \fixme{Maybe rephrase}.

In chapter 4, we discuss the methods used to examine the problem and evaluate solutions we will propose. In chapter 5, we describe the methods used to solve the problem. In chapter 6, we evaluate the implementation and its results. \fixme{Chapters 7 and 8 missing. Also, maybe elaborate description about chapters 4-6 a bit}

\fixme{After writing each chapter, check the description in this section}

% You should use transition in your text, meaning that you should help
% the reader follow the thesis outline. Here, you tell what will be in
% each chapter of your thesis. 

