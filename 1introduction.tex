%!TEX root = thesis.tex

\chapter{Introduction}
\label{chapter:intro}

The introduction in itself is rarely very long; two to five pages often
suffice.

~

\fixme{Here maybe something related to why I chose this topic, i.e. why it is beneficial for me/others to have a framework that eases the map visualization process.}

Geographical data is data with geospatial dimension, such as POI with location data as coordinates. The most natural method for visualizing geographical data is usually with various maps. In the past, geographical data was predominantly visualized by cartographers, but it has been recognized \citep{kraak_visualization_1999} that the situation has changed, with people from increasing number of fields having a need for visualizing geographical data. Moreover, the popularity of Google Maps \citep{google_maps_2005} along with its API \citep{google_maps_2005-1} has proved that in addition to experts of other academic fields, there is a definite demand for web map visualizations within consumers as well. 

The web makes publishing and bundling map visualizations extraordinarily straightforward when compared to traditional desktop-based GIS applications, which is especially important when the visualizations are made by non-cartographers \fixme{find a reference for this maybe}. However, as the web platform is primarily designed for static documents instead of dynamic applications \citep{berners-lee_information_1989,berners-lee_world-wide_1992}, there are some additional concerns to address when making a complex data visualization on the web.

\fixme{Why geodata should be visualized? \citet[chap.~1.1]{kraak_cartography_2011} or \citet{bartz_petchenik_place_1979} should help.}

\fixme{Needs more flesh}

\section{Problem Statement}

Currently, there are several libraries available for displaying maps and simple visualizations \fixme{add references to the libraries. GMaps, Leaflet, OpenLayers etc.}. However, none of the mainstream libraries is of sufficiently high abstraction level for building map visualizations effectively, resulting in the need for writing \emph{boilerplate} code that does not directly contribute to the visualization. Moreover, the libraries are not designed primarily for visualizations and therefore do not encourage or push the visualizer to create visually and cognitively effective visualizations.

We plan to evaluate means to make creating map visualizations for the web more efficient by building a higher abstraction level software framework for map visualizations. This framework should provide the structure for creating the visualization as well as common web application features needed in modern web applications.

In order to find the solution for the problem, it is necessary to study geographical visualizations and software frameworks. The process of making geographical visualizations should be studied to ensure that the framework enables creating \emph{effective} visualizations. In addition, software frameworks and software reuse should be studied in order to be able to create visualizations \emph{efficiently}. Therefore, we select the following research questions for this thesis:

\begin{enumerate}
	\item[RQ1] How to build effective geographical visualizations in the web?
	\item[RQ2] How to build a web software framework which enables creating effective geographical visualizations efficiently?
	% old one:
	% \item[RQ2] How to measure software framework efficiency reliably?

\end{enumerate}
\fixme{Would it make sense to drop the RQ1 since RQ2 pretty much covers it? Or rephrase the questions to overlap less?}

% TODO remove
% \section{Helpful hints}

% Read the information from the university master's thesis
% pages~\citep{ThesisInstructions} before starting the thesis.  You
% should also go through the thesis grading
% instructions~\citep{ThesisGrading} together with your instructor and/or
% supervisor in the beginning of your work.

\section{Structure of the Thesis}
\label{section:structure} 

Chapter 1 (this introduction) presents the motivation for this thesis as well as the problem statement. Chapter 2 describes the background of the work. In particular, the chapter describes how to visualize geographical data and the essence of web software frameworks. Chapter 3 presents the web technology, standards and other needed material for building a web visualization as well as describing some existing map visualizations \fixme{Maybe rephrase}.

In chapter 4, we discuss the methods used to examine the problem and evaluate solutions we will propose. In chapter 5, we describe the methods used to solve the problem. In chapter 6, we evaluate the implementation and its results. \fixme{Chapters 7 and 8 missing. Also, maybe elaborate description about chapters 4-6 a bit}

\fixme{After writing each chapter, check the description in this section}

% You should use transition in your text, meaning that you should help
% the reader follow the thesis outline. Here, you tell what will be in
% each chapter of your thesis. 

