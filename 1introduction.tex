%!TEX root = thesis.tex

\chapter{Introduction}
\label{chapter:intro}

% The introduction in itself is rarely very long; two to five pages often
% suffice.

We are confronted with a quickly increasing amount of data every day. We also increasingly need to use the data as a basis for our actions and thoughts. Data visualization enables us to obtain insight about data quickly and efficiently \citep{van_wijk_value_2005}, making it crucial in the modern world. 

An estimated 95 \% of all digital data contains geographical references \citep{perkins_have_2010}. Visualizing this data helps users perceive geospatial relationships and patterns. Additionally, maps can be used for determining information on distances, directions and areas \citep[chap.~1.1]{kraak_cartography_2011}. Using geographical visualizations, it is also possible to organize data spatially and visually, allowing more efficient memorization of data.

Geographical data is data with geospatial dimension, such as Point of Interest (POI) with location data as coordinates \citep[chap.~1.2]{kraak_cartography_2011}. The most natural method for visualizing geographical data is usually with various maps. In the past, geographical data was predominantly visualized by cartographers, but it has been recognized \citep{kraak_visualization_1999} that the situation has changed, with people from increasing number of fields having a need -- and the possibility \citep[chap.~1]{slocum_thematic_2014} -- for visualizing geographical data. Moreover, the popularity of Google Maps \citep{google_maps_2005-1} along with its Application Programming Interface (API) \citep{google_maps_2005} has proved that in addition to experts of other academic fields, there is a definite demand for web map visualizations within consumers as well. 

The web makes publishing and bundling map visualizations extraordinarily straightforward when compared to traditional desktop-based Geographic Information System (GIS) applications: traditional desktop-based GIS system requires an installation of the GIS application and often additional tools and accounts for publishing the visualization, while using web-based mapping software ideally requires no additional software or tools or even accounts. This is especially important when the visualizations are made by non-cartographers who only make visualizations occasionally and lack the needed resources and experience for more complex publishing process \citep{miller_beast_2006}. However, as the web platform is primarily designed for static documents \citep{berners-lee_information_1989,berners-lee_world-wide_1992} instead of dynamic applications \citep{jazayeri_trends_2007}, there are some additional concerns to address when making a complex data visualization on the web.

\section{Problem Statement}

Currently, there are several libraries available for displaying maps and simple visualizations \citep{google_maps_2005,agafonkin_leaflet_2011,metacarta_openlayers_2006}. However, the problem is that none of the mainstream libraries is of sufficiently high abstraction level for building map visualizations efficiently, resulting in the need for writing \emph{boilerplate} code that does not directly contribute to the visualization. Moreover, the libraries are not designed primarily for visualizations and therefore do not encourage or push the visualizer to create visually and cognitively effective visualizations, resulting in subpar visualizations \citep[chap.~1]{slocum_thematic_2014}.

In the scope of this thesis, we adopt the process definitions of \citet{van_wijk_value_2005} by making the difference between an efficient and an effective process. By an efficient process, we mean a process which requires as little as possible effort and other resources to complete. By an effective process, we mean a process which reaches its objectives sufficiently. Therefore, building a visualization efficiently indicates that the building process is as effortless as possible, and building effective visualization indicates that the resulting visualization conveys its intended message appropriately.

\section{Objectives and Scope}

Our primary objective is to make creating map visualizations for the web more efficient by building a reusable higher abstraction level software system for map visualizations, and to evaluate the efficiency benefits of the system. This system should provide the structure for creating the visualization as well as common web application features needed in modern web applications. Our secondary objective is to encourage more effective visualizations by considering the cognitive requirements of visualizations when building the system.

In order to find the solution for the problem, it is necessary to study geographical visualizations and software reuse. The process of making geographical visualizations should be studied to ensure that the system encourages creating \emph{effective} visualizations. In addition, software reuse should be studied in order to be able to create versatile visualizations \emph{efficiently}. Therefore, to evaluate the objectives, we select the following research questions for this thesis:

\begin{enumerate}
	\item[RQ1] How does a reusable software system affect the \emph{efficiency} of building geographical visualizations?
	\item[RQ2] How does a reusable software system affect the \emph{effectiveness} of geographical visualizations?
\end{enumerate}

% a) how does a reusable software system affect the efficiency of ...
% b) can a reusable software system...


\section{Approach}

In order to build an efficient system for visualizing geographical data, it is needed to study (a) how to visualize geographical data and (b) how to build reusable software. We begin by first studying the basics of data visualization with an emphasis on geographical data, maps and the visualization process. After visualization, we study the essence of software reuse, focusing on building and evaluating reusable software. Based on suggestions from earlier research, we proceed to create a reusable visualization tool and evaluate its effect on visualization effectiveness and efficiency of the building process.

\section{Structure of the Thesis}
\label{section:structure} 

Chapter \ref{chapter:intro} (this introduction) presents the motivation for this thesis as well as the problem statement. Chapter \ref{chapter:reuse} presents the essence of software reuse, concentrating on success factors and methods for reuse. Chapter \ref{chapter:dataviz} describes the fundamentals of data visualizations with an emphasis on geographic data and thematic mapping. In chapter \ref{chapter:reuseinvisualization}, we discuss the research on data visualization reuse along with its shortcomings, and argue about the need for this research.

In chapter \ref{chapter:methods}, we present some of the most prominent methods for evaluating software reuse and visualization effectiveness, selecting the most suitable methods for this work. In chapter \ref{chapter:implementation}, we describe the implementation of the tool designed to address the shortcomings presented in chapter \ref{chapter:reuseinvisualization}. In chapter \ref{chapter:evaluation}, we evaluate the implementation based on the methods presented in chapter \ref{chapter:methods} and analyze the evaluation results. In chapter \ref{chapter:discussion}, we interpret the evaluation results along with their applicability, shortcomings and generalizability. We also propose topics for further research based on this work. In chapter \ref{chapter:conclusions}, we conclude the findings and other implications of the thesis.