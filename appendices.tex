%!TEX root = thesis.tex

\chapter{First appendix}
\label{chapter:first-appendix}

This is the first appendix. You could put some test images or verbose data in an
appendix, if there is too much data to fit in the actual text nicely.

For now, the Aalto logo variants are shown in Figure~\ref{fig:aaltologo}.

\begin{figure}
\begin{center}
\subfigure[In English]{\includegraphics[width=.8\textwidth]{images/aalto-logo-en}}
\subfigure[Suomeksi]{\includegraphics[width=.8\textwidth]{images/aalto-logo-fi}}
\subfigure[P� svenska]{\includegraphics[width=.8\textwidth]{images/aalto-logo-se}}
\caption{Aalto logo variants}
\label{fig:aaltologo}
\end{center}
\end{figure}

\chapter{Flat Dot Format}
\label{appendix:flatdotformat}

Flat dot format is a simple, but non-standard format used by a number of web mapping applications such as the Store Finder of Alko\footnote{\url{http://www.alko.fi/myymalat/}}. Format consists of a JSON file representing an array of zero or more objects. The objects must contain latitude and longitude properties, and may contain a number of other properties. An example of the format is depicted below.

\begin{lstlisting}[language=json]
[
  {
    "_id": "51b9f04d27eaa49f4a0001cc",
    "number": 2,
    "name": "Destination",
    "latitude": 60.314322,
    "longitude": 24.554067
  },
  {
    "_id": "51b9f0f3127ceac8db000263",
    "number": 0,
    "name": "Departure",
    "latitude": 60.314041,
    "longitude": 24.551678
  },
  {
    "_id": "51b9fa600752f83f720003c3",
    "number": 1,
    "name": "Pit stop",
    "latitude": 60.316474,
    "longitude": 24.556554
  }
]
\end{lstlisting}