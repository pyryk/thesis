%!TEX root = thesis.tex

\chapter{Methodology}
\label{chapter:methods}

\section{Evaluating Software Reuse Effectiveness}

Several methods for analyzing software reuse exist. \citet{frakes_software_1996} present six general types of software reuse analysis models: cost-benefit analysis models, maturity assessment models, amount-of-reuse models, failure modes analysis, reusability assessment models and reuse library metrics. 

Cost-benefit analysis models consider both development costs for the reusable component and the reuse productivity and quality benefits. Maturity assessment models categorize the matureness of a systematic software reuse program. Amount of reuse metrics assess the proportion of reused software in the software system created. Software reuse failure modes model introduces a set of reuse failures which are then used to assess a systematic reuse program. Reusability assessment models aim to analyze various software attributes to assess the reusability of a piece of code. Reuse library metrics concentrate on the reusability of software component libraries instead of single components.

Not all the models discussed above are applicable to this type of research. For instance, as maturity assessment models assess a reuse program as a whole instead of single reusable piece of software, it can not be used for this type of work. Moreover, it should be noted that these models are high-level analysis tools which do not take a stance on how to measure the detailed data required by the measurements.

In their review of multiple case studies, \citet{mohagheghi_quality_2007} list several methods as alternatives for analyzing software reuse. Of them, notable ones include \emph{controlled experiments}, \emph{case studies}, \emph{surveys} and \emph{experience reports}. However, as the scope of this work does not allow for large sample sizes required by controlled experiment method, it is not a feasible alternative for this study. Moreover, using experience reports requires a considerable experience on creating and using reusable software.

According to \citet{kitchenham_evaluating_1998}, one method for conducting a case study is using a \emph{sister project} comparison. In sister project comparison, a minimum of two different, but sufficiently similar software projects observed. In at least one of the projects, a new method is employed, while in at least one project, the old method is in use. All other practices and aspects should be left unchanged. \citet{mohagheghi_quality_2007} argue that this kind of comparison is applicable for analyzing software reuse effectiveness for a specific piece of software. The sister project case study is appropriate when no systematic reuse program exists or when there are other barriers impeding the use of models presented by \citet{frakes_software_1996}. Moreover, it is possible to create the sister project synthetically by building the same kind of application twice, once with and once without the reusable component.

As presented above, in the higher abstraction level, analyzing projects is fairly straightforward. However, the actual low-level measurements for reusing software are much more complex. \citet{mohagheghi_quality_2007} argue that measuring software reuse effectiveness precisely is difficult due to a number of factors: 1) Metrics are difficult to validate since there is no universally accepted definition of ``quality'' in software products, and 2) the productivity of development is difficult to measure and therefore highly subjective, vague or even erroneous metrics are utilized.

Both \citet{frakes_software_1996} and \citet{mohagheghi_quality_2007} agree that the size of the code needed to be written correlates inversely to the development productivity. Moreover, research by \citet{banker_software_1993,gill_cyclomatic_1991} show that software code complexity correlates inversely to the software maintenance productivity. Due to the nature of software maintenance and the fact that it is often hard to distinguish small-scale software development and maintenance \citep{chapin_types_2001}, it is likely that this principle applies to developing new software to some degree as well. Therefore, it seems evident that in order to enable productive use of reusable software, the new code to be written (outside the reusable components) should be made simple and concise.

The conciseness of the code can be measured by several different metrics. According to \citet{fenton_software_1998}, the number of lines in program source code is only one perspective. It can be complemented by measuring the functionality and complexity of the program.

According to \citet{fenton_software_1998}, the most commonly used metric for program size is its length, i.e., the number of lines of source code. The metric can be refined by only considering effective lines, ignoring lines consisting of comments and whitespace. However, \citet{fenton_software_1998} encourages the use of both effective and physical line counts to determine the size of a program.

The functionality of the program can be determined with several different metrics. \citet{fenton_software_1998} presents the function point approach, which uses the number and complexity of external inputs and outputs, object point approach which uses the number of different screens and reports involved in the application, and so-called ``bang metrics'' which use the total number of primitives in the data-flow diagram of the program. It should be noted that all of these methods are highly subjective and only provide speculative metrics.

Also the complexity of the program can be determined with several different techniques. According to \citet{fenton_software_1998}, the complexity of the program (solution) should ideally be not higher than the problem complexity. According to them, in the ideal case it is possible to determine the complexity of the program by determining the complexity of the program. However, this is not usually the case in real-life applications. \citet{mccabe_complexity_1976} presents a computational approach for determining the complexity of an application implementation. His approach is used by counting the number of cycles in the program flow graph. The advantage of this approach when compared to the method presented by \citet{fenton_software_1998} is that it can be used to compute complexity differences of multiple implementations of the problem.

\fixme{Halstead measurements? Esp. effort}

\section{Research Method Chosen for the Analysis}

For this work, we considered both the methods presented by \citet{frakes_software_1996} and \citet{mohagheghi_quality_2007}. As stated in the previous section, several of the methods are only applicable for assessing large-scale reuse programs or cases with large sample size and therefore were not considered for this work. Of the remaining methods, a cost-benefit analysis was selected. For studying costs and benefits of the reuse, a case study with sister project comparison approach was deemed most applicable. 

For comparing the projects, software size and complexity metrics by \citet{fenton_software_1998} were used, with the exception that measuring software complexity is done with the method by \citet{mccabe_complexity_1976} as it allows comparing different implementations of similar applications. \fixme{Add specifics about tools, restrictions etc. See chapter 7.1 of achy}

For gathering data, we implemented three \fixme{confirm the number} map visualizations with and without the framework. The first visualization is a simple use case which concentrates on the display of a custom map. The second visualization highlights POI data on a map, and the third one is the most complex, consisting of POI data, relations between POIs and a custom backend serving the data with real-time updates from a mobile client. The types of visualizations were selected to obtain data about a wide variety of different map visualizations. \fixme{Probably the visualizations ought to a bit more complex than this}

\fixme{How are the measurements combined? (1st viz without framework + 2nd viz without framework + 3rd wiz without framework) <=> (1st viz with framework + 2nd viz with framework + 3rd wiz with framework + framework)? Or should the framework be excluded?}

% TODO move to implementation? or have some "how we use the method in practice" section here?



% You have now stated your problem, and you are ready to do something
% about it!  \emph{How} are you going to do that? What methods do you
% use?  You also need to review existing literature to justify your
% choices, meaning that why you have chosen the method to be applied in
% your work.

% An example of a traditional LaTeX table
% ------------------------------------------------------------------
% A note on underfull/overfull table cells and tables:
% ------------------------------------------------------------------
% In professional typography, the width of the text in a page is always a lot
% less than the width of the page. If you are accustomed to the (too wide) text
% areas used in Microsoft Word's standard documents, the width of the text in
% this thesis layout may suprise you. However, text in a book needs wide
% margins. Narrow text is easier to read and looks nicer. Longer lines are 
% hard to read, because the start of the next line is harder to locate when
% moving from line to the next. 
% However, tables that are in the middle of the text often would require a wider
% area. By default, LaTeX will complain if you create too wide tables with
% ``overfull'' error messages, and the table will not be positioned properly
% (not centered). If at all possible, try to make the table narrow enough so
% that it fits to the same space as the text (total width = \textwidth).
% If you do need more space, you can either
% 1) ignore the LaTeX warnings 
% 2) use the textpos-package to manually position the table (read the package
%    documentation)
% 3) if you have the table as a PDF document (of correct size, A4), you can use
%    the pdfpages package to include the page. This overrides the margin
%    settings for this page and LaTeX will not complain.
% ------------------------------------------------------------------
% Another note:
% ------------------------------------------------------------------
% If your table fits to \textwidth, but the cells are so narrow that the text
% in p{..}-formatted cells does not flow nicely (you get underfull warnings 
% because LaTeX tries to justify the text in the cells) you can manually set
% the text to unjustified by using the \raggedright command for each cell 
% that you do not want to be justified (see the example below). \raggedleft 
% is also possible, of course...
% ------------------------------------------------------------------
% If you need to have linefeeds (\\) inside a cell, you must create a new
% paragraph-formatting environment inside the cell. Most common ones are 
% the minipage-environment and the \parbox command (see LaTeX documentation
% for details; or just google for ``LaTeX minipage'' and ``LaTeX parbox'').
\begin{table}
\begin{tabular}{|p{2cm}|p{3.8cm}|p{4.5cm}|p{1.1cm}|} 
% Alignment of sells: l=left, c=center, r=right. 
% If you want wrapping lines, use p{width} exact cell widths.
% If you want vertical lines between columns, write | above between the letters
% Horizontal lines are generated with the \hline command:
\hline % The line on top of the table
\textbf{Code} & \textbf{Name} & \textbf{Methods} & \textbf{Area} \\ 
\hline 
% Place a & between the columns
% In the end of the line, use two backslashes \\ to break the line,
% then place a \hline to make a horizontal line below the row 
T-110.6130 & Systems Engineering for Data Communications
    Software & \raggedright Computer simulations, mathematical modeling,
  experimental research, data analysis, and network service business
  research methods, (agile method) & T-110 \\ 
\hline
\multicolumn{2}{|p{6.25cm}|}{Mat-2.3170 Simulation (here is an example of
 multicolumn for tables)}& Details of how to build simulations & T-110 \\
% The multicolumn command takes the following 3 arguments: 
% the number of cells to merge, the cell formatting for the new cell, and the
% contents of the cell
\hline
S-38.3184 & Network Traffic Measurements and Analysis 
& \raggedright How to measure and analyse network
  traffic & T-110 \\ \hline
\end{tabular} % for really simple tables, you can just use tabular
% You can place the caption either below (like here) or above the table
\caption{Research methodology courses}
% Place the label just after the caption to make the link work
\label{table:courses}
\end{table} % table makes a floating object with a title

