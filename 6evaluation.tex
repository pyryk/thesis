%!TEX root = thesis.tex

\chapter{Evaluation}
\label{chapter:evaluation}

In this chapter, we describe the evaluation of the tool built. The evaluation is performed by using two separate methods: we evaluate the efficiency of development process with the software effort and complexity metrics presented in chapter \ref{chapter:methods}, and visualization effectiveness with principles and guidelines by \citet{tufte_visual_1986}, \citet{azzam_j-b_2013} and \citet{kraak_cartographic_1998} presented in chapter \ref{section:visualizationprinciples}. As using either of the methods requires a baseline project, we decided to implement a number of sister projects as defined by \citet{kitchenham_evaluating_1998}.

\section{Defining the Evaluated Cases}

The visualization tool should be able to visualize a large variety of data. Moreover, the benefits of reusable software are typically emphasized when examining a large number of relatively similar cases \citep{frakes_software_1996}. However, in order to keep the scope of this work manageable, we decided to evaluate a set of visualization cases listed below. While the visualized data is arbitrarily selected, the cases are picked to reflect the typical usage of visualizations.

\begin{itemize}
	\item 1 \textbf{dot map visualization}: locations of Alko stores in Finland
	\item 1 \textbf{proportional symbol map visualization}: \fixme{what to visualize?}
	\item 3 \textbf{choropleth map visualizations}: regional circulation of the biggest Finnish newspapers
	\item 1 \textbf{dasymetric map visualization}: \fixme{what to visualize?}
	\item \fixme{Add Isarithmic? Then would cover all methods}
\end{itemize}

We selected a different number of cases for some mapping methods. This was done to reflect the approximate relative use frequency of methods; choropleth map is the most frequenly used thematic mapping method \citet[chap.~14]{slocum_thematic_2014}, while dot map and proportional symbol maps are 

In order to better model typical real-life use cases, and to be usable on the web, the visualization cases include also a generic application structure and HTML features \fixme{such as...} which are required by a web application.

\section{Implementing Sister Projects}

\section{Evaluating Efficiency of Development}

\section{Evaluating Effectiveness of Visualizations}

\begin{itemize}
	\item Principles by \citet{tufte_visual_1986} (data ink, dont distort, etc.)
	\item Truthfulness by \citet{azzam_j-b_2013}
	\item Guidelines in \citet{kraak_cartographic_1998} (how do i say what to whom and is it effective)
\end{itemize}

You have done your work, but that's\footnote{By the way, do \emph{not} use
shorthands like this in your text! It is not professional! Always write out all
the words: ``that is''.} not enough. 

You also need to evaluate how well your implementation works.  The
nature of the evaluation depends on your problem, your method, and
your implementation that are all described in the thesis before this
chapter.  If you have created a program for exact-text matching, then
you measure how long it takes for your implementation to search for
different patterns, and compare it against the implementation that was
used before.  If you have designed a process for managing software
projects, you perhaps interview people working with a waterfall-style
management process, have them adapt your management process, and
interview them again after they have worked with your process for some
time. See what's changed.

The important thing is that you can evaluate your success somehow.
Remember that you do not have to succeed in making something spectacular; a
total implementation failure may still give grounds for a very good master's
thesis---if you can analyze what went wrong and what should have been done.

 
